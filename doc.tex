\documentclass[utf8]{ctexart}

% preli start
\usepackage{amssymb}
\usepackage{amsmath}
\usepackage{mathalfa}
\usepackage{algorithm}
\usepackage[noend]{algpseudocode}

\makeatletter
\def\BState{\State\hskip-\ALG@thistlm}
\makeatother
% preli end


\begin{document}

\title{Parameter estimation for text analysis - 笔记}
\author{冯柏淋}
\date{2018年10月23日}

\maketitle
\tableofcontents
\newpage

\section{引入}
此笔记是本人在阅读论文Gregor Heinrich:\textit{Parameter estimation for text analysis}的过程的一点笔记,大多是对于原文的翻译和摘录。另有一部分自己的理解,因而可能存在错误。

\section{基本问题与贝叶斯公式}

\subsection{两类基本问题}

\begin{itemize}
	\item	\textbf{估计} \, 估计分布的参数以解释观测值 
	\item	\textbf{预测/回归} \, 基于以往观测值,预测新的观测值的概率 
\end{itemize}

\subsection{贝叶斯公式}
\begin{equation}\label{eq1}
p(\theta \mid \mathcal{X} ) = \frac{p(\mathcal{X} \mid \theta) \cdot p(\theta)}{p(\mathcal{X})}
\end{equation}

对以上公式中的每一部分,我们有如下名称:
\begin{equation}\label{eq2}
\text{posterior} = \frac{\text{likelihood} \cdot \text{prior}}{\text{evidence}}
\end{equation}

\subsection{记号}
数据集:$$	\mathcal{X} \triangleq \lbrace x \rbrace _{i}^{|\mathcal{X}|}	$$

\section{极大似然估计MLE}

\subsection{参数估计}
极大似然函数:
\begin{equation}\label{eq3}
L(\theta ; \mathcal{X}) \triangleq p(\mathcal{X} \mid \theta) = \prod_{x \in \mathcal{X}} p(x \mid \theta)
\end{equation}

极大似然估计(MLE):
\begin{equation}\label{eq4}
\hat{\theta}_{\text{MLE}} = \arg \max_\theta \mathcal{L}(\theta ; \mathcal{X}) =  \arg \max_\theta \sum_{x\in \mathcal{X}} \log p(x \mid \theta)
\end{equation}

\subsection{预测}
对于新的观测值的概率估计过程如下:\footnote{此处,$p(\tilde{x} \mid \hat{\theta}_{\text{MLE}})$视为常数,并且$\int_{\theta \in \Theta} p(\theta \mid \mathcal{X}) \, \text{d} \theta = 1$}

\begin{align}
p(\tilde{x} \mid \mathcal{X}) 
	&= \int_{\theta \in \Theta} p(\tilde{x} \mid \theta) \cdot p(\theta \mid \mathcal{X}) \, \text{d} \theta \\
	&\approx \int_{\theta \in \Theta} p(\tilde{x} \mid \hat{\theta}_{\text{MLE}}) \cdot p(\theta \mid \mathcal{X}) \, \text{d} \theta  \\
	&= p(\tilde{x} \mid \hat{\theta}_{\text{MLE}}).
\end{align}


\subsection{一个例子}
暂时略。

\section{最大后验概率MAP}
\subsection{参数估计}
\begin{align}
\hat{\theta}_{\text{MAP}} &= \arg \max_\theta \, p(\theta \mid \mathcal{X})	\\
	&= \arg \max_\theta \, \frac{p(\mathcal{X} \mid \theta) \cdot p(\theta)}{p(\mathcal{X})} \\
	&= \arg \max_\theta \, p(\mathcal{X} \mid \theta) \cdot p(\theta) \\
	&= \arg \max_\theta \, \lbrace  \mathcal{L}(\theta \mid \mathcal{X}) + \log p(\theta)  \rbrace \\
	&= \arg \max_\theta \, \lbrace \sum_{x \in \mathcal{X}} \log p(x \mid \theta) + \log p(\theta) \rbrace
\end{align}

MAP相比于MLE而言,添加了关于先验分布的额外信息$p(\theta)$。实践中,可以通过假定一些简单的先验分布$p(\theta)$来防止过拟合。\footnote{Occam's razor原理}

\subsection{预测}

\begin{align}
p(\tilde{x} \mid \mathcal{X}) 
	&\approx \int_{\theta \in \Theta} p(\tilde{x} \mid \hat{\theta}_{\text{MAP}}) \cdot p(\theta \mid \mathcal{X}) \, \text{d} \theta  \\
	&= p(\tilde{x} \mid \hat{\theta}_{\text{MAP}}).
\end{align}

\subsection{一个例子}
暂时略。

\section{贝叶斯估计}
\subsection{参数估计}
贝叶斯估计是对MAP的一个拓展。MAP对于参数只产生一个确定的估计值$\hat{\theta}_{\text{MAP}}$,而贝叶斯估计则产生对于参数的一个分布。从而,贝叶斯分布则提供了关于参数的额外信息,比如期望和方差。

主要步骤为计算后验分布:
\begin{equation}
	p(\theta \mid \mathcal{X}) = \frac{p(\mathcal{X} \mid \theta) \cdot p(\theta)}{p(\mathcal{X})}
\end{equation}

在前述的MAP方法中,只需要最大化此式,则无需求出分母$p(\mathcal{X})$,而在此处则需要计算,且:
\begin{equation}
	p(\mathcal{X}) = \int_{\theta \in \Theta} p( \mathcal{X} | \theta ) \cdot  p( \theta ) \, \text{d} \theta 
\end{equation}
因为没最大化参数的过程,参数估计的过程也就不是一个具体的值,而是求取参数的分布:
\begin{align}
	p(\theta \mid \mathcal{X}) &= \frac{p(\mathcal{X} \mid \theta) \cdot p(\theta)}{p(\mathcal{X})}	\\
	&=  \frac{p(\mathcal{X} \mid \theta) \cdot p(\theta)}{ \int_{\theta \in \Theta} p( \mathcal{X} | \theta ) \cdot  p( \theta ) \, \text{d} \theta }
\end{align}

\subsection{预测}

\begin{align}
p(\tilde{x} \mid \mathcal{X}) 
	&= \int_{\theta \in \Theta} p(\tilde{x} \mid \theta) \cdot p(\theta \mid \mathcal{X}) \, \text{d} \theta \\
	&= \int_{\theta \in \Theta} p(\tilde{x} \mid \theta) \cdot \frac{p(\mathcal{X} \mid \theta) \cdot p(\theta)}{p(\mathcal{X})} \, \text{d} \theta .
\end{align}

\subsection{一个例子}
暂时略。

\end{document}